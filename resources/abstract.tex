\documentclass[../main.tex]{subfiles}

\begin{document}

\section*{Abstract}
\markboth{Abstract}{}  % To create a phantom chapter "Abstract" for header formatting

Abstract goes here.

This is temporary, but look back to this as a reference while writing the report. In the first class lecture, the instructures laid out the grading scheme and a rough grading rubric for the report:

\begin{table}[h!]
    \centering
    \begin{tabular}{|c|c|}
        \hline
        \textbf{Section} & \textbf{Points} \\
        \hline
        Abstract & 5 \\
        Introduction & 10 \\
        Literature Review & 5 \\
        Methodology & 15 \\
        Results & 15 \\
        Conclusion & 5 \\
        Bibliography & 5 \\
        \hline
    \end{tabular}
    \caption{Grading rubric.}
    \label{tab:abstract:grading_rubric}
\end{table}

Following this rubric, I think we should split our project into the following sections

\begin{enumerate}
    \item Abstract (DO THIS LAST)
    \item Introduction
    \item Research Statement
    \item Literature Review
    \item Clustering Algorithm
    \begin{itemize}
        \item Input data description; rationale for "why" we picked this
        \item HCA description (variables, cutting, etc.)
        \item Linkage method discussion
        \begin{itemize}
            \item Discuss each linkage method
        \end{itemize}
        \item Discussion of candidate sectors; why we need validation model to "rank" sector universes
    \end{itemize}
    \item Validation Methodology
    \begin{itemize}
        \item Use candidate universe to explain "universe by universe" comparison
        \item reIndexer
        \begin{itemize}
            \item Overview
            \item Software architecture
            \item synthetic ETF construction
            \item minimum variance portfolio computation
        \end{itemize}
        \item Target performance metrics (portfolio/etf construction costs, sharpe, etc.)
        \item Backtest configuration
    \end{itemize}
    \item Ranking results (discuss individual best-ones here too; refer back to validation methodology)
    \item Conclusion
    \item Future Work (deep-ish dive into 2/3 possible avenues)
\end{enumerate}

\end{document}